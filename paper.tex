\documentclass{article}

% Macro to generate neighborhood memory map table
\newcommand\memtable[1]{
\begin{tabular}{lll}
  \hline
  From & To & Contents \\
  \hline
  #1
  \hline
\end{tabular}
}

\newcommand\memrow[4]{
    {\tt {#1}00} & {\tt {#1}FF} & State of cell $(x#2,y#3,z#4)$ \\
}

% The \memxyz macro gives a densely-connected (von Neumann) neighborhood
\newcommand\memz[5]{
  \memrow{#1}{#4}{#5}{-1}
  \memrow{#2}{#4}{#5}{}
  \memrow{#3}{#4}{#5}{+1}
}

\newcommand\memyz[2]{
  \memz{{#1}0}{{#1}1}{{#1}2}{#2}{-1}
  \memz{{#1}4}{{#1}5}{{#1}6}{#2}{}
  \memz{{#1}8}{{#1}9}{{#1}A}{#2}{+1}
}

\newcommand\memxyz{
  \memyz{4}{-1}

  %  \memyz{5}{} includes zero page-mapped cell (x,y,z), which we want to skip, so spell it out without that
  \memz{50}{51}{52}{}{-1}
  \memrow{54}{}{}{-1}
  \memrow{56}{}{}{+1}
  \memz{58}{59}{5A}{}{+1}

  \memyz{6}{+1}
}

% The following table is a sparse neighborhood
\newcommand\vonneumannmap{\memtable{
  \memrow{45}{-1}{}{}
  \memrow{51}{}{-1}{}
  \memrow{54}{}{}{-1}
  \memrow{56}{}{}{+1}
  \memrow{59}{}{+1}{}
  \memrow{65}{+1}{}{}
}}

% The following neighborhood is dense
\newcommand\mooremap{\memtable{\memxyz}}

% Document
\begin{document}

\title{Infinite Cellular Automata Running Virtual 6502 Machines}
\author{Ian Holmes \\ Department of Bioengineering, University of California, Berkeley, USA}

\maketitle


\begin{abstract}
  Cellular automata.
  6502s.
  Blockchain.
\end{abstract}

\section{Cellular automata}

The cellular automaton is a $L \times L \times L$ array of cells.
Cells are indexed $(x,y,z$) where $0 \leq x \leq L$, $0 \leq y \leq L$, $0 \leq z \leq L$.
Each cell has 256 bytes (one ``page'') of state.
All cells' bytes are initially zero.

Cells are updated in a pseudorandom order.
This can be any order that visits all cells uniformly.
If $L$ is a power of 2, $L = 2^N$ for some integer $N$,
we can use the simple ordering $0 \leq i \leq 2^{3N}$
with $(x_i,y_i,z_i)$ defined such that
the $k$'th bit of $x_i$ is the $3N-3k-1$'th bit of $i$,
the $k$'th bit of $y_i$ is the $3N-3k-2$'th bit of $i$, and
the $k$'th bit of $z_i$ is the $3N-3k-3$'th bit of $i$
(using LSB-0 numbering).

A cell $(x,y,z)$ is updated by emulating an 8-bit virtual machine.
The CPU is the MOS Technology 6502 processor.
The memory map is shown in \ref{tab:MemoryMap},
with detail for the memory-mapping of adjacent cells in \ref{tab:MooreNeighborhood}.

\begin{table}
\begin{tabular}{llll}
  \hline
  From & To & Type & Contents \\
  \hline
  {\tt 0000} & {\tt 00FF} & RAM (storage-mapped) & State of cell $(x,y,z)$ \\
  {\tt 0100} & {\tt 01FF} & RAM (transient) & Stack, initialized to zero \\
  {\tt 0200} & {\tt 03FF} & RAM (transient) & ``Scratch'' memory, initialized to zero \\
  {\tt 0400} & {\tt 0401} & ROM (emulator-mapped) & Cycles left before termination \\
  {\tt 0402} & {\tt 0402} & ROM (emulator-mapped) & Random byte generator \\
  {\tt 0400} & {\tt 3FFF} & ROM & Reserved, zero \\
  {\tt 4000} & {\tt 7FFF} & RAM (storage-mapped) & Memory-mapped neighborhood (Table~\ref{tab:MooreNeighborhood}) \\
  {\tt 8000} & {\tt FEFF} & ROM & Read-only (operating system) \\
  {\tt FF00} & {\tt FFFF} & ROM & Includes interrupt and reset vectors. All zero \\
  \hline
\end{tabular}
\caption{
  \label{tab:MemoryMap}
  Memory map for the 6502 virtual machine of cell $(x,y,z)$.
  Note that the BREAK vector {\tt FFFC-FFFD} is zero, so execution begins at {\tt 0000}.
}
\end{table}

\begin{table}
\mooremap
\caption{
  \label{tab:MooreNeighborhood}
  Memory map for the 6502 virtual machine of cell $(x,y,z)$,
  using the 26-cell three-dimensional Moore neighborhood.
}
\end{table}

\begin{table}
\vonneumannmap
\caption{
  \label{tab:VonNeumannNeighborhood}
  Memory map for the 6502 virtual machine of cell $(x,y,z)$,
  using the 6-cell three-dimensional von Neumann neighborhood.
}
\end{table}

All other bytes are read-only and zero; writing to them has no effect.
This can be understood as the cell $(x-1+i,y-1+j,z-1+k)$
mapping to page number (64+i*16+j*4+k) with $i,j,k \in \{0,1,2\}$.
Thus, in a memory-mapped 16-bit address, the bits have the following meanings

\begin{tabular}{ll}
  \hline
  Bit(s) & Meaning \\
  \hline
  14,15 & Must be {\tt 01} \\
  12,13 & $i$ \\
  10,11 & $j$ \\
  8,9 & $k$ \\
  0-7 & Address within page \\
  \hline
\end{tabular}

Note the exception: cell $(x,y,z)$ is not mapped to {\tt 5500-55FF}
because it is mapped to zero page {\tt 0000-00FF} instead.




\section{Blockchain description}

\cite{Nakamoto2008}

\section*{Acknowledgments}
Thanks to Chris Evans, Richard Evans, Michael Mateas.

\bibliographystyle{plain}
\bibliography{references}

\end{document}
